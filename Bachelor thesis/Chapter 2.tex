\chapter{Резонансни решения}

    Съществуват редица методи за контрол на квантови системи с две и повече нива, които често намират приложение в различни практически области като например атомната оптика, лазерния контрол на химичи реакции, т.нар. оптични центруфуги,
    квантовата информатика и др. В следващите няколко глави (2, 3 и 5) ще се спрем на метод, познат в литературата като адиабатен квантов контрол на системи с две и три нива. Тук ще започнем с най-простият случай на резонансни системи като
    постепенно ще налагаме допълнителни условия до достигане на решения, в които вече липсва резонанс поради наличието на т.нар. честотни отмествания.\\
    Нека разгледаме система с две нива, при която степента на заселеност за всяко ниво се променя с помощта на външно поле. Хамилтонианът система-поле ще се задава с уравнението:
    
    \section{Системи с две нива}

    \begin{equation}
        \hat{H} = \hat{H}_0 - \hat{\textbf{d}}\cdot \textbf{E}(t),
    \end{equation}

    където $\hat{H}_0$ е Хамилтониана на изолираната система, а $\hat{\textbf{d}}\cdot \textbf{E}(t)$ е частта, описваща взаимодействието на системата с външното поле. Интензитетът на полето удовлетворява уравнението:
    
    \begin{equation}
        \textbf{E}(t) = \textbf{$\epsilon$} E_0 \cos(\omega t).
    \end{equation}

    Тук \textbf{$\epsilon$} е векторът на поляризация, $E_0$ амплитудата, а $\omega$ честотата на приложеното поле.\\
    Използвайки теорията на нестационарните пертурбации получаваме системата диференциални уравнения: 
    
    \begin{subequations}
        \begin{align}
            & \textit{i}\frac{d}{dt}c_1(t) = \frac{d_{12}E_0}{\hbar}e^{-\textit{i}\omega_{21} t}\cos(\omega t)c_2(t),\\
            & \textit{i}\frac{d}{dt}c_2(t) = \frac{d_{12}E_0}{\hbar}e^{\textit{i}\omega_{21} t}\cos(\omega t)c_1(t),
        \end{align}
    \end{subequations}

    където $d_{12}$ е матричният елемент $\bra{1}\hat{\textbf{d}}\cdot\textbf{$\epsilon$}\ket{2}$ или математическото очакване на диполния момент $\bra{1}\hat{\textbf{d}}\ket{2}$, проектирано върху вектора на поляризация \textbf{$\epsilon$},
    а $\Omega=\frac{d_{12}E_0}{\hbar}$ има размерност на честота и се нарича честота на Раби. Използвайки тригонометричното съотношение $\cos(\omega t) =\frac{1}{2} \left(e^{\textit{i}\omega t}+e^{-\textit{i}\omega t}\right)$ преобразуваме
    диференциалните уравнения във вида:
    
    \begin{subequations}
        \begin{align}
            & \textit{i}\frac{d}{dt}c_1(t) = \frac{\Omega}{2}e^{-\textit{i}\omega_{21} t}\left(e^{\textit{i}\omega t}+e^{-\textit{i}\omega t}\right)c_2(t),\\
            & \textit{i}\frac{d}{dt}c_2(t) = \frac{\Omega}{2}e^{\textit{i}\omega_{21} t}\left(e^{\textit{i}\omega t}+e^{-\textit{i}\omega t}\right)c_1(t),
        \end{align}
    \end{subequations}

    или след умножаване на експонентите:
    
    \begin{subequations}
        \begin{align}
            & \textit{i}\frac{d}{dt}c_1(t) = \frac{\Omega}{2}\left(e^{\textit{i}(\omega -\omega_{21}) t}+e^{-\textit{i}(\omega +\omega_{21}) t}\right)c_2(t),\\
            & \textit{i}\frac{d}{dt}c_2(t) = \frac{\Omega}{2}\left(e^{-\textit{i}(\omega -\omega_{21}) t}+e^{\textit{i}(\omega +\omega_{21}) t}\right)c_1(t).
        \end{align}
    \end{subequations}

    Членът в експонентата $\Delta=\omega -\omega_{21}$ задава честотното отместване на полето спрямо Боровата честота на системата (детюнинг). Другият член в експонентата $\omega +\omega_{21}$
    предизвиква много бързи осцилации и при сравнително бавно-изменящи се вектори на състоянието можем да приложим т.нар. приближение на въртящата се вълна (Rotating Wave Approximation)
    и да пренебрегнем високочестотните членове:
    
    \begin{subequations}
        \begin{align}
            & \textit{i}\frac{d}{dt}c_1(t) = \frac{\Omega}{2}e^{\textit{i}(\omega -\omega_{21}) t}c_2(t),\\
            & \textit{i}\frac{d}{dt}c_2(t) = \frac{\Omega}{2}e^{-\textit{i}(\omega -\omega_{21}) t}c_1(t).
        \end{align}
    \end{subequations}

    Това приближение е валидно само в случаите на нулев или близък до нула детюнинг ($\Delta \approx 0$).\\
    В настоящата задача поради наличието на резонанс честотата на полето и Боровата честота са равни, което означава $\Delta = 0$ и системата придобива най-простия възможен вид:
    
    \begin{subequations}
        \begin{align}
            & \textit{i}\frac{d}{dt}c_1(t) = \frac{\Omega}{2}c_2(t),\\
            & \textit{i}\frac{d}{dt}c_2(t) = \frac{\Omega}{2}c_1(t).
        \end{align}
    \end{subequations}

    Тази система се решава чрез диференциране на първото уравнение по времето:
    
    \begin{equation}
        \textit{i}\frac{d^2}{dt^2}c_1(t) = \frac{\Omega}{2}\frac{d}{dt}c_2(t)
    \end{equation}

    и заместване на производната $\frac{d}{dt}c_2(t)$ с нейното равно от второто уравнение. Това води до следното диференциално уравнение от втори ред:
    
    \begin{equation}
        \frac{d^2}{dt^2}c_1(t) =-\frac{\Omega^2}{4}c_1(t).
    \end{equation}

    Решението на последното диференциално уравнение е:
    
    \begin{equation}
        c_1(t) = \cos\left(\frac{\Omega t}{2}\right),
    \end{equation}

    удовлетворяващо началните условия: $c_1(0) = 1$ и $c_2(0) = 0$.
    По аналогичен начин получаваме решението за втората компонента $c_2(t)$:

    \begin{equation}
        c_2(t) = -\textit{i}\sin\left(\frac{\Omega t}{2}\right).
    \end{equation}

    За да получим стойностите на заселеностите просто взимаме абсолютната стойност на всяка компонента на \textbf{c}(t) и повдигаме на квадрат:
    
    \begin{subequations} \label{2.12}
        \begin{align}
            P_1 = |c_1(t)|^2 = \cos^2\left(\frac{\Omega t}{2}\right),\\
            P_2 = |c_2(t)|^2 = \sin^2\left(\frac{\Omega t}{2}\right),
        \end{align}
    \end{subequations}

    т.е. заселеностите и за двете нива осцилират във времето по синусоидален закон. Това поведение се нарича осцилации на Раби:

    \begin{figure}[H]
        \centering
        \includegraphics[width=310pt]{19.png}
    \end{figure}
    \begin{center}
        \small \textit{Фигура 8: Графика на заселеността на система с две нива за време t. Решението на горната задача води до осцилации, познати като осцилации на Раби. За определеност честотата на Раби
        $\Omega = 2$.}
    \end{center}
    
    Уравнения (\ref{2.12}) могат да бъдат допълнително преобразувани във вида:
    \begin{subequations} \label{2.1}
        \begin{align}
            P_1 = \frac{1}{2}\left(1+\cos\Omega t\right),\\
            P_2 = \frac{1}{2}\left(1-\cos\Omega t\right).
        \end{align}
    \end{subequations}

    Тези решения са в сила за постоянно поле. Когато амплитудата на полето варира, аргумента на косинуса се заменя с така наречената площ на импулса (pulse area):

    \begin{equation}
        \Omega t \rightarrow \int_{-\infty}^{t}\Omega(t')dt' = \mathcal{A}(t)
    \end{equation}

    От (\ref{2.1}) става ясно че заселеността при резонанс осцилира между състоянията 0 и 1, като какво ще бъде крайното състояние зависи от стойността на $\mathcal{A}(t)$.
    Ако площта на импулса е равна на четни стойности на $\pi$, системата остава в началното състояние (състоянието в което се е намирал осцилатора преди прилагане на външно поле) и
    обратно, ако $\mathcal{A}(t)$ е равна на нечетни стойности на $\pi$, системата претърпява пълен популационен преход от едното в другото състояние.\\
    Интересни са случаите когато $\mathcal{A}(t) = \pi$ или $\mathcal{A}(t) = \frac{\pi}{2}$. В първия случай заселеността на първоначално незаселеното ниво достига $P = 1$, което съответства на
    пълна размяна на заселените състояния или т.нар. X-Gate. Във втория случай, когато площта на импулса е равна на $\frac{\pi}{2}$, максималната стойност на заселеността е $P = \frac{1}{2}$,
    което съответства на суперпозиция от състояния в края на еволюцията. Това е познато в литературата като гейт на Адамард (Hadamard Gate) \nocite{vitanov2010quantum}.

    \section{Системи с три нива}

    Сега ще разгледаме поведението на малко по-сложните системи с три нива, при които вече Хамилтонианът представлява матрица 3х3 от вида:

    \begin{equation} \label{5.1}
        \hat{H}(t) = \hbar
        \begin{bmatrix}
            0 & \frac{1}{2}\Omega_1(t) & 0 \\
            \frac{1}{2}\Omega_1(t) & \Delta(t) & \frac{1}{2}\Omega_2(t) \\
            0 & \frac{1}{2}\Omega_2(t) & \delta(t)
        \end{bmatrix},
    \end{equation}

    където $\Omega_1(t)$ и $\Omega_2(t)$ представляват честотите на Раби при преход между първо и второ състояние и второ и трето състояние, а
    $\Delta(t)$ и $\delta(t)$ са съответните честотни разлики.\\
    Ако $\delta = 0$, настъпва двуфотонен резонанс и системата преминава директно от първо в трето състояние. Хамилтонианът на системата добива вида:
    \begin{equation}
        \hat{H}(t) = \hbar
        \begin{bmatrix}
            0 & \frac{1}{2}\Omega_1(t) & 0 \\
            \frac{1}{2}\Omega_1(t) & \Delta(t)-\textit{i}\Gamma & \frac{1}{2}\Omega_2(t) \\
            0 & \frac{1}{2}\Omega_2(t) & 0
        \end{bmatrix}.
    \end{equation}
    Тогава уравнението на Шрьодингер за амплитудите на вероятност е:
    \begin{equation} \label{5.3}
        \textit{i}\frac{d}{dt}\begin{bmatrix}
            c_1\\
            c_2\\
            c_3
        \end{bmatrix} = \begin{bmatrix}
            0 & \frac{1}{2}\Omega_1(t) & 0 \\
            \frac{1}{2}\Omega_1(t) & \Delta(t)-\textit{i}\Gamma & \frac{1}{2}\Omega_2(t) \\
            0 & \frac{1}{2}\Omega_2(t) & 0
        \end{bmatrix} \, \begin{bmatrix}
            c_1\\
            c_2\\
            c_3
        \end{bmatrix}.
    \end{equation}
    Нека $\Omega_1(t)$ и $\Omega_2(t)$ са положителни, тъй като както стана ясно в Глава 3 заселеностите не зависят от знаците им. Също така често е по-удобно $\Omega_1(t)$ и
    $\Omega_2(t)$ да споделят една и съща функционална зависимост (обаче с различни амплитуди):
    \begin{equation} \label{5.4}
        \Omega_1(t) = A_1g(t) \quad , \quad \Omega_2(t) = A_2g(t).
    \end{equation}
    Нека предположим и че началната стойност $\textbf{c}(-\infty) = [c_1(-\infty), c_2(-\infty), c_3(-\infty)]^T$ е известна. Тогава задачата се свежда до намирането на
    заселеностите $P_\textit{i} = \left|c_\textit{i}(+\infty)\right|^2$ за нивата $\textit{i} = 1, 2, 3$.\\
    При решаването на подобен тип задачи е удобно да се въведат т.нар. тъмно и светло състояния. Тъмното представлява суперпозиция от състоянията $\ket{1}$ и $\ket{3}$. Липсата
    на $\ket{2}$ го прави нечувствително към разпадане на метастабилното средно състояние към други по-ниски нива, а понякога може дори да се отдели от полетата и изобщо да не
    взаимодейства с тях. Нека дефинираме светлото $\ket{b}$ и тъмното $\ket{d}$ състояния по следния начин:

    \begin{subequations}
        \begin{align}
            \ket{b} = \sin\theta\ket{1} + \cos\theta\ket{3},\\
            \ket{d} = \cos\theta\ket{1} - \sin\theta\ket{3}.
        \end{align}
    \end{subequations}

    Тук ъгълът $\theta$ се задава чрез тригонометричното равенство:

    \begin{equation}
        \tan\theta = \frac{\Omega_1(t)}{\Omega_2(t)} = \frac{A_1}{A_2}.
    \end{equation}

    Понеже функционалната зависимост от времето за импулсите е една и съща (уравнение (\ref{5.4})), $\theta$ е константна, а следователно и $\ket{b}$ и $\ket{d}$. Прилагайки трансформацията:

    \begin{equation} \label{5.7}
        \begin{bmatrix}
            c_1\\
            c_2\\
            c_3
        \end{bmatrix} = \begin{bmatrix}
            \sin\theta & 0 & \cos\theta\\
            0 & 1 & 0\\
            \cos\theta & 0 & -\sin\theta
        \end{bmatrix} \, \begin{bmatrix}
            c_b\\
            c_2\\
            c_d
        \end{bmatrix},
    \end{equation}

    уравнението (\ref{5.3}) се преобразува до:

    \begin{equation} \label{5.8}
        \textit{i}\frac{d}{dt}\begin{bmatrix}
            c_b\\
            c_2\\
            c_d
        \end{bmatrix} = \begin{bmatrix}
            0 & \frac{1}{2}\Omega_0(t) & -\textit{i}\dot{\theta} \\
            \frac{1}{2}\Omega_0(t) & \Delta(t)-\textit{i}\Gamma & 0 \\
            \textit{i}\dot{\theta} & 0 & 0
        \end{bmatrix} \, \begin{bmatrix}
            c_b\\
            c_2\\
            c_d
        \end{bmatrix},
    \end{equation}

    където сме положили $\Omega_0(t) = \sqrt{\Omega_1^2(t) + \Omega_2^2(t)} = \sqrt{A_1^2 + A_2^2}g(t)$. Производната $\dot{\theta} = 0$, тъй като $\theta = const$, което отделя състояние $\ket{d}$
    от другите две и равенство (\ref{5.8}) се трансформира в задача за двете нива $\ket{b}$ и $\ket{2}$:

    \begin{equation}
        \textit{i}\frac{d}{dt}\begin{bmatrix}
            c_b\\
            c_2
        \end{bmatrix} = \begin{bmatrix}
            0 & \frac{1}{2}\Omega_0(t)\\
            \frac{1}{2}\Omega_0(t) & \Delta(t)-\textit{i}\Gamma
        \end{bmatrix} \, \begin{bmatrix}
            c_b\\
            c_2
        \end{bmatrix}.
    \end{equation}

    От последното равенство веднага се вижда връзката на състояние $\ket{2}$ с $\ket{b}$ чрез $\Omega_0$. Отделянето на тъмното състояние води до запазване на неговата заселеност:

    \begin{equation}
        P_d(t) = P_d(-\infty) = \left|c_1(-\infty)\cos\theta - c_3(-\infty)\sin\theta\right|^2,
    \end{equation}

    а тъй като тази заселеност не може нито да бъде прехвърлена на състояние $\ket{2}$, нито да излезе от системата чрез някакъв вид разпад, неравенството

    \begin{equation}
        1-P_1(t) - P_3(t) \le 1-P_d(-\infty)
    \end{equation}

    остава в сила за всеки момент от прехода. При по-подробно разглеждане на последното съотношение се вижда, че съществува горна граница на заселеността на второто ниво при $\Gamma = 0$
    ($P_2 = 1 - P_1 - P_3$), както и за популационните загуби при $\Gamma > 0$ ($P_{loss} = 1 - P_1 - P_3$).\\
    Нека пропагаторът, дефинирен чрез уравнението
    \begin{equation} \label{5.12}
        \left[c_b(+\infty),c_2(+\infty)\right]^T = \mathbf{U}^{(2)} \left[c_b(-\infty),c_2(-\infty)\right]^T
    \end{equation}
    е зададен по следния начин:

    \begin{equation}
        \textbf{U}^{(2)} = \begin{bmatrix}
            a & b\\
            c & d
        \end{bmatrix}.
    \end{equation}

    Тогава за пропагатора на система с три нива в базиса $\ket{b} \, \ket{d}$ ще имаме:

    \begin{equation} \label{5.14}
        \textbf{U}_{bd}^{(3)} = \begin{bmatrix}
            a & b & 0\\
            c & d & 0\\
            0 & 0 & 1
        \end{bmatrix}
    \end{equation}

    Използвайки уравнения (\ref{5.7}) и (\ref{5.14}) за явния вид на матрицата на еволюция получаваме:

    \begin{equation} \label{5.15}
        \textbf{U}^{(3)} = \begin{bmatrix}
            a\sin^2\theta + cos^2\theta & b\sin\theta & (a-1)\sin\theta\cos\theta\\
            c\sin\theta & d & c\cos\theta\\
            (a-1)\sin\theta\cos\theta & b\cos\theta & a\cos^2\theta + \sin^2\theta
        \end{bmatrix}
    \end{equation}
    
    Нека видим как изглеждат изразите за заселеностите $P_1 \, , P_2$ и $P_3$ на системата за някое начално условие (да кажем $\textbf{c}(-\infty)=(1,0,0)^T$).
    Използвайки началната стойнст на вектора на състоянието и формули (\ref{5.12}) за три нива и (\ref{5.15}) за заселеностите на всяко от нивата в края на еволюцията
    получаваме следните равенства:
    \begin{subequations}
        \begin{align}
            & P_1(+\infty) = \left|a\sin^2\theta+\cos^2\theta\right|^2,\\
            & P_2(+\infty) = \left|c\right|^2\sin^2\theta,\\
            & P_3(+\infty) = \left|a-1\right|^2\sin^2\theta\cos^2\theta.
        \end{align}
    \end{subequations}
    По аналогичен начин можем да получим заселеностите за всяка друга стойност на $\textbf{c}(-\infty)$. При по-задълбочен анализ се вижда, че единственият необходим
    праметър за определяне на заселеностите на системата е параметърът $a$. В изразите за заселеностите не са включени останалите параметри $b,c$ и $d$, а само
    техните модули, които са свързани с $a$ по различен начин в зависимост от стойността на константата на разпад $\Gamma$. При $\Gamma = 0$ имаме взаимоотношенията
    $|d|=|a|$ и $|b|^2=|c|^2=1-|a|^2$, а за $\Gamma>0 \rightarrow b=c=d=0$.

    \vspace{5mm}

    Подобно на резултатите в глава 2 за система с три нива също съществува точен резонанс (еднофотонен резонанс). Един пример е система с три нива, при която едно след
    друго се прилагат две полета -- напомпващо и Стоксово. Това силно наподобява процесът SEP (Stimulated Emission Pumping) за две нива, като системата с три нива може
    да се разгледа като последователност от два прехода за система с две нива. Въпреки това вероятностите са различни за двата вида системи. За системата с три нива
    вероятностите $P_{1\leftrightarrow 2}$ и $P_{2\leftrightarrow 3}$ са равни на:
    \begin{subequations}
        \begin{align}
            P_{1\leftrightarrow 2} = \frac{1}{2}\left(1-\cos A_P\right)\\
            P_{2\leftrightarrow 3} = \frac{1}{2}\left(1-\cos A_S\right),
        \end{align}
    \end{subequations}
    като вероятността за директен преход между състояния $\ket{1}$ и $\ket{3}$ се задава с произведението на горните две:
    \begin{equation}
        P_{1\leftrightarrow 3} = \frac{1}{4}\left(1-\cos A_P\right)\left(1 - \cos A_S\right).
    \end{equation}
    Тук $A_P$ и $A_S$ са съответните стойности на площта на импулса за пулсовото и Стоксовото поле.\\
    Ако напомпващото и Стоксово полета имат една и съща зависимост от времето, динамиката на процеса вече не може да се разглежда като два последователни процеса на
    система с две нива. Въпреки това могат да се получат решения и в този случай. Например заселеността в състояние $\ket{3}$ в края на прехода $\ket{1}\rightarrow\ket{3}$
    ще се задава с формулата:
    \begin{equation} \label{5.19}
        P_3 = \frac{A_PA_S}{A}\left(1-\cos\frac{1}{2}A\right),
    \end{equation}
    където $A = \sqrt{A_P^2+A_S^2}$. От равенство (\ref{5.19}) се вижда, че когато двете стойности на $A_P$ и $A_S$ са равни и $A = 2(2k+1)\pi$ $(k=0,1,2,\dots)$
    заселеността преминава напълно от състояние $\ket{1}$ в състояние $\ket{3}$. При $A=4k\pi$ заселеостта се връща обратно в състояние $\ket{1}$.