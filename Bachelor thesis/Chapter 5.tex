    \chapter{Адиабатни решения за системи с три нива}

    \section{Известни методи за контрол с пресичане на нивата}

    Съществуват експерименти, в които са се осъществявали контролирани преходи в системи с три нива като тези, описани в статии \cite{broers1992efficient} и \cite{melinger1992generation}.
    В първата за система с три нива се използват рубидиеви атоми, като се преминава последователно през 5s $\rightarrow$ 5p $\rightarrow$ 5d атомни орбитали,
    след прилагане на последователност от два бавно променящи се по отношение на честотата (frequency sweep) лазерни импулса. Това води до пълен пренос на заселеността
    от 5s към 5p, а накрая 5d атомна орбитала. 

    \begin{figure}[H]
        \centering
        \includegraphics[width=430pt]{3-Level.png}
    \end{figure}
    \begin{center}
        \small \textit{Фигура 19: Енергитична стълбица за рубидиев атом. Петият слой на електронната му обвивка може да се използва като система с три нива,
        при която електронът прескача от една орбитала в друга след прилагане на външно поле с подходяща честота. \textbf{A)} при нормална последователност
        на приложените полета заселеността преминава изцяло от най-ниско към най-високо, като в средата на процеса заселеността е на второто ниво.
        \textbf{Б)} при противоинтуитивна последователност на полетата заселеността прескача директно на най-високото ниво, пропускайки средното.}\\
    \end{center}

    Интересно е да се отбележи, че при прилагане на обратната последователност от импулси (т.нар. противоинтуитивна последователност)
    системата не само осъществява преход от 5s към 5d, но това се случва директно, без преминаване през състоянието 5p. Такъв метод разбира се е интересен в случаите когато
    средното ниво е нежелано заради фактори като нестабилност на заселването на средното състояние или сравнително бързо отдаване на заселеност от него към други нива в системата.

    \begin{figure}[H]
        \centering
        \includegraphics[width=270pt]{11.png}
    \end{figure}
    \begin{center}
        \small \textit{Фигура 20: Схематично представяне на прехода на заселеността между трите състояния в системата. Със стрелки и цветен градиент са означени посоката и вида
        на приложеното поле. Правата посока (червено към синьо) води до преход 5s $\rightarrow$ 5p $\rightarrow$ 5d. Противоинтуитивната посока на приложеното поле
        (синьо към червено) води до директен преход 5s $\rightarrow$ 5d.}\\
    \end{center}

    В другият експеримент натриеви пари се облъчват за много кратко време (пикосекунди) със същият бавно-изменящ се лазерен импулс, предизвикващ инверсия и селективно заселване на
    нивата. За основно състояние се използва 3s нивото на натриевия атом, а като възбудени се вземат натриевия дублет - Фраунхоферовите линии за нивото 3p на натрия $(^2P_{1/2} \quad ; \quad ^2P_{3/2})$.
    Измерването се осъществява чрез прехода на едно от възбудените нива към ниво 5s. В зависимост от посоката на промяна на честотата на лазерното поле е установено, че се извършва преход или от основно
    към $^2P_{1/2}$ (червено $\rightarrow$ синьо), или от основно към $^2P_{3/2}$ (синьо $\rightarrow$ червено). Преходът се осъществява чрез процес, известен като бърз адиабатен преход (ARP), подпомагащ
    селективността и устойчивостта на процеса. ARP е много слабо чувствителен към честотата или формата на амплитудата на приложения лзерен импулс, стига промяната на последния да е достатъчно бавна, за да
    удовлетвори условието за адиабатен преход: $\left|\frac{d\theta}{dt}\right|\ll \sqrt{\Omega^2 + \Delta^2} \quad ; \quad \theta = \arctan\left(\frac{\Delta}{\Omega}\right)$.

    \begin{figure}[H]
        \centering
        \includegraphics[width=450pt]{25.png}
    \end{figure}
    \begin{center}
        \small \textbf{(a)} \textit{Фигура 21: Заселеност на нивата $^2S_{1/2} (\ket{1}) , \, ^2P_{1/2} (\ket{2})$ и $^2P_{3/2} (\ket{3})$ за натриев атом. Честотата на Раби на използвания лазерен импулс значително надвишава
        разстоянието между двете Фраунхоферови линии $(\approx 17 \, cm^{-1})$. \textbf{(b)} Енергиите на състоянията по време на прехода. В зависимост от посоката на промяна на честотата на лазерното поле може да
        бъде избрано определено крайно състояние ($\ket{2}$ или $\ket{3}$).\\
        Източник: \href{https://journals.aps.org/prl/abstract/10.1103/PhysRevLett.68.2000}{Phys. Rev. Lett. 68, 2000}} \cite{melinger1992generation}\\
    \end{center}

    Представените по-горе доказани експериментални методи могат да послужат като основа при формирането на нови подходи за квантов контрол. На база на тези идеи сега ще разгледаме елиминирането на нежелани квантови състояния в системи с три нива.

    \section{Елиминиране на нежелани квантови състояния в адиабатни квантови гейтове на свръхпроводников кюбит}
    
    Дотук разгледахме случаите на едно- и двуфотонен резонанс в системи с три нива при които детюнинга е равен на нула. Нека сега разгледаме адиабатните решения, при които
    $\Delta,\delta\ne 0$, и системата вече се състои не от две (както в глава 3), а от три енергетични нива. В зависимост от разположението на последните системите с три нива
    се делят на три основни класа: стълбица, ламбда $(\Lambda)$ и ве $(V)$.

    \begin{figure}[H]
        \centering
        \includegraphics[width=270pt]{20.jpg}
    \end{figure}
    \begin{center}
        \small \textit{Фигура 22: Схематично представяне на трите възможни вида преходи в система с три нива, в зависимост от разположението на енергетичните нива.}\\
    \end{center}

    Това, което ще разгледаме по-точно е евентуалната възможност за реализиране на кюбит в система с три нива. Въпросът, който трябва да си зададем е могат ли да се подберат такива
    контролни параметри, които да изолират първите две нива от третото, като в същото време системата не е в някой от резонансните случаи, разгледани по-горе.\\
    Можем да си отговорим като проведем някои числени симулации. Нека нашата система се описва с Хамилтониана:

    \begin{equation} \label{5.20}
        \hat{H} = \begin{bmatrix}
            0 && \frac{1}{2}\Omega(t) && 0\\
            \frac{1}{2}\Omega(t) && \Delta(t) && \frac{1}{\sqrt{2}}\Omega(t)\\
            0 && \frac{1}{\sqrt{2}}\Omega(t) && \delta +2\Delta(t)
        \end{bmatrix},
    \end{equation}
    
    който води до следната система обикновени диференциални уравнения:

    \begin{subequations} \label{5.21}
        \begin{align}
            & \textit{i}\frac{d}{dt}c_1(t) = \frac{\Omega}{2}c_2(t),\\
            & \textit{i}\frac{d}{dt}c_2(t) = \frac{\Omega}{2}c_1(t) + \Delta c_2(t) + \frac{\Omega}{\sqrt{2}} c_3(t),\\
            & \textit{i}\frac{d}{dt}c_3(t) = \frac{\Omega}{\sqrt{2}} c_2(t) + (\delta + 2\Delta) c_3(t).
        \end{align}
    \end{subequations}

    Ще разгледаме отново еволюцията на системата при прилагане на X и Hadamard Gate, описани за системи с две нива в глава 3. Ще започнем с X Gate. За него честотата на Раби $\Omega$ и
    детюнинга $\Delta$ се задават чрез функциите:

    \begin{equation} \label{5.22}
        \Omega(t) = \Omega_0 e^{-\frac{(t-\tau)^2}{T^2}} \quad , \quad \Delta(t) = \alpha t
    \end{equation}

    където $\Omega_0$ е коефициента на Раби импулса (директно свързан с адиабатността на прехода), $\tau$ е тайминга на Раби импулса (в кой момент от прехода прилагаме
    външното поле), $T^2$ е ширината на Раби импулса, а $\alpha$ е коефициентът на линейност на функцията $\Delta$. Разликата между приложените честоти на полето
    при преход между всяка двойка нива е специфична за всяка отделна система. За удобство нека положим $\delta \rightarrow 1$.\\
    При подбиране на параметрите:
    \begin{equation} \label{5.23}
        \Omega_0 = 0,62 \,\,\, ; \,\,\, T = 3,28 \,\,\, ; \,\,\, \tau = 1,3 \,\,\, ; \,\,\, \alpha = 0,1
    \end{equation}
    и числено решаване на системата диференциални уравнения (\ref{5.21}), се осъществява изисквания преход, а именно пълно обръщане на заселеността между състоянията $\ket{0}$ и $\ket{1}$ с минимално влияние, оказвано от страна на състояние $\ket{2}$.
    Решението, заедно с наложените параметри води до $P_1(+\infty) \rightarrow 0 \, ;\,\, P_2(+\infty) \rightarrow 0,99 \, ;\,\, P_3(+\infty) \rightarrow 0,01$.

    \begin{figure}[H]
        \centering
        \includegraphics[width=370pt]{23.png}
    \end{figure}
    \begin{center}
        \small \textit{Фигура 23: Графика, показваща разпределенето по заселености между нивата за X Gate.}\\
    \end{center}

    Нека сега разгледаме еволюцията на системата от гледна точка на собствените енергии на Хамилтониана (\ref{5.20}). От графиката по-долу се вижда, че системата
    преминава първоначално през два диабатни прехода, които не водят до промяна на местата на някое от енергетичните нива. Последният преход обаче е адиабатен и
    води до обръщане на местата на основното състояние с първото възбудено (точно каквото се очаква при реализиране на X Gate).

    \begin{figure}[H]
        \centering
        \includegraphics[width=300pt]{26.png}
    \end{figure}
    \begin{center}
        \small \textit{Фигура 24: Енергитична диаграма, показваща еволюцията на системата с три нива за определените по-горе параметри. Виждат се диабатните/адиабатните преходи
        за всяко пресичане на енергетичните нива (червения пунктир)}\\
    \end{center}

    Задачата за получаването на Hadamard Gate е подобна на описаната преди малко, като честотното отместване се задава по следния начин:
    \begin{equation}
        \Delta(t) = \beta \left(\tanh(\alpha t)-1\right),
    \end{equation}
    а честотата на Раби има същата функционална зависимост като за X Gate (\ref{5.22}).\\
    След решаване на системата диференциални уравнения (\ref{5.21}) за параметрите:

    \begin{equation}
        \Omega_0 = 0.4\, ;\,\,T = 2.325\, ;\,\,\tau = 4\, ;\,\,\alpha = 0.5\, ;\,\,\beta = 2
    \end{equation}

    за заселеностите в края на прехода на всяко ниво получаваме:
    \begin{equation}
        P_1(+\infty) \rightarrow 0,49 \, ;\,\, P_2(+\infty) \rightarrow 0,49 \, ;\,\, P_3(+\infty) \rightarrow 0,2,
    \end{equation}

    т.е. в края на процеса основното и първото възбудено ниво се намират в състояние на суперпозиция, като всяко притежава
    около половината от първоначалната заселеност.

    \begin{figure}[H]
        \centering
        \includegraphics[width=370pt]{24.png}
    \end{figure}
    \begin{center}
        \small \textit{Фигура 25: Графика, представяща популационния пренос при Hadamard Gate.}\\
    \end{center}

    Отново, ако разгледаме еволюцията на системата от гледна точка на собствените енергии на Хамилтониана (\ref{5.20}), ще
    видим, че системата претърпява два диабатни прехода, които в края успещно отделят второто възбудено ниво от другите две.

    \begin{figure}[H]
        \centering
        \includegraphics[width=300pt]{27.png}
    \end{figure}
    \begin{center}
        \small \textit{Фигура 26: Фигура, представяща еволюцията на гореописаната система с три нива за Hadamard Gate.}\\
    \end{center}

    \section{Свойства на адиабатните квантови гейтове}

    Разгледаните тук адиабатни преходи за системи с три нива притежават някои предимства пред други методи за квантов контрол
    като различни устойчивости на външни влияния и известни промени в параметрите на системата. Както е показано на графиките
    по-долу, отнасящи се за операцията X Gate, при сравнително големи вариации на параметрите на външното поле системата запазва
    крайните стойности на заселеност на отделните нива, а именно почти сто процентова заселеност на първото възбудено ниво и почти
    нулева за останалите две.\\
    На фигурите по-долу могат да се видят и някои интересни резултати като например наличието на асиметрия на графиката, показваща заселеността на
    първото възбудено ниво като функция на детюнинга. Такъв тип асиметрия е характерен именно за $n > 2$, където $n = 1, 2, 3, \dots$ е броя на нивата в системата.

    \begin{figure}[H]
        \centering
        \includegraphics[width=350pt]{29.png}
    \end{figure}
    \begin{center}
        \small \textit{Фигура 27: Графика на заселеността като функция на детюнинга (параметъра $\alpha$) с характернта за системи с повече от две нива
        асиметрия спрямо нулата.}
    \end{center}

    На графиката за заселеността като функция на тайминга се вижда рязко понижаване на крайната заселеност, което започва при $\tau \approx 6$. Това може да се дължи на
    отварянето на нежелан кросинг, през който да изтича част от заселеността на системата. Тук също има наличие на известна асиметрия, която може да се обясни с промяната
    в топологията на системата (Фигура 24). За различен момент от време $t$, в системата съществуват различни по вид пресичания на нивата.

    \begin{figure}[H]
        \centering
        \includegraphics[width=350pt]{30.png}
    \end{figure}
    \begin{center}
        \small \textit{Фигура 28: Заселеността спрямо тайминга ($\tau$).}
    \end{center}

    За заселеността спрямо продължителността на Раби импулса също се наблюдава спад на крайната заселеност на желаното ниво, но тук този процес е по-бавен от разгледания за
    тайминга. Това поведение може отново да се обясни с наличието на нежелани кросинги по време на действието на импулса, като при голяма продължителност вероятността да се
    попадне на такова пресичане става все по-голяма. За разлика от предишните две разглеждания тук се наблюдава пълна симетрия по отношение на нулата, тъй като при решаването
    на задачата за заселеностите на нивата използваме т.нар. ширина на Раби импулса ($T^2$).

    \begin{figure}[H]
        \centering
        \includegraphics[width=350pt]{31.png}
    \end{figure}
    \begin{center}
        \small \textit{Фигура 29: заселеността като функция на продължителността ($T$) на Раби импулса.}
    \end{center}

    За последната графика, показваща зависимостта на заселеността на крайното състояние спрямо честотата на Раби, могат да се направят аналогични на горните разсъждения. Тук също
    се наблюдава понижаване на крайната заселеност при симетрична промяна на стойността на $\Omega_0$ спрямо нулата.

    \begin{figure}[H]
        \centering
        \includegraphics[width=350pt]{28.png}
    \end{figure}
    \begin{center}
        \small \textit{Фигура 30: Заселеността спрямо честотата на Раби.}
    \end{center}

    Този тип методи притежават и някои недостатъци като например ниска скорост на протичане на процесите (породена от условието за адиабатност)
    и изключително трудно постигане на висока точност за контролиране на системата. Въпреки това за процеси, за които времетраенето или прецизността на контрол
    не са от съществено значение, такъв тип адиабатни преходи са изключително подходящи като средство за квантов контрол.