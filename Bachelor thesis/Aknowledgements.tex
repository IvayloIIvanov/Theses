\chapter{Заключение}

В тази дипломна работа бяха представени накратко основите на адиабатните преходи като средство за контрол
на кюбитите в квантови системи, основно тези на IBM. Описани са основите на конструирането на т.нар. свръхпроводникови
квантови компютри, както и теорията зад адиабатните преходи за системи с две нива и за системи с три нива при наличие на
резонанс. Тези методи представляват интерес най-вече заради своята устойчивост на външни влияния и малки промени в контролните
параметри, което ги прави особено подходящи за системи, в които изчислителните единици не са много добре изолирани. Беше разгледан и нов, алтернативен
метод за създаване на кюбит, контролиран чрез адиабатен преход. Новото в този подход е, че се разглежда не вече добре познатата система
с две нива, а система с три нива, при която третото, най-високо енергитично ниво е изолирано от останалите две и по време на
еволюцията на системата не получава почти никаква заселеност от основното състояние. Като потвърждение на това
твърдение бяха показани две числени симулации на X и Hadamard гейтове, при които се вижда поведение, силно наподобяващо
това на система, в която третото ниво изобщо не присъства. Тези резултати водят до предположението, че подобна система
може да бъде използвана като алтернатива на добре изучените системи с две нива, както и евентуално наличие на системи с
повече от три нива, които да могат да се свеждат успешно до системи с две нива при правилен подбор на контролните параметри
в техния Хамилтониан.