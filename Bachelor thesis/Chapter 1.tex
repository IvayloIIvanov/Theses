\chapter{Въведение}

    \normalsize Квантовите компютри представляват интерес от страна на научната общност и индустрията от десетилетия, не само заради възможността
    да бъдат използвани законите на квантовата механика за практическа реализация на съвсем нов тип изчислителен апарат,
    но и, както се оказва в някой случаи, устройство с много по-голяма изчислителна мощ от познатите класически компютри
    днес. За разлика от класическия бит, който може да заема две основни стойности (0 и 1) изчислителната единица на квантовия
    компютър -- кюбитът може да се намира в състояние на суперпозиция между двете, като след точно определена последователност от
    операции (прилагане на външно поле и използване на правилни алгоритми за контрол) се достига до желаното състояние. Това позволява
    използването на нови методи за решаване на сложни задачи, които биха отнели години изчисления дори и на най-мощния класически компютър.
    Такива задачи са например моделирането на сложни молекули за създаването на нови материали с определени свойства, точни симулации представящи
    взаимодействията между заредени частици (електрони и йони) в големи системи като органични молекули, симулиране на йонния транспорт в различни
    видове батерии и други.
    Решенията, разгледани тук намират приложение в т.нар. свръхпроводникови квантови компютри. Тяхното устройство и принцип на действие
    ще бъдат обяснени накратко по-долу.

    \section{Принцип на действие на свръхпроводников квантов компютър}
    
    Свръхпроводниковия квантов компютър използва свръхпроводникови електрически елементи (като материали се използват например ниобий и алуминий) за реализиране на своята изчислителна
    дейност, наречени трансмони. Те се изграждат с помощта на кондензатори и връзки на Джозевсън. Последните представляват два свръхпроводника, разделени
    от слой с по-ниска проводимост или изолатор. При подаване на напрежение между свръхпроводниците, протича т.нар. свръхток, който продължава
    да тече дори след спиране на напрежението, поради липса на съпротивление. Тази комбинация кондензатор + връзки на Джозевсън са предпочитани при конструирането на
    свръхпроводникови квантови устройства, тъй като са пример за проява на квантови ефекти на макро ниво (връзките на Джозевсън се държат като нелинейни
    индукционни елементи, които в комбинация с кондензатор проявяват свойства на ахармоничен осцилатор), което позволява по-лесен контрол над състоянията в системата.\\

    \begin{figure}[H]
        \centering
        \includegraphics[width=370pt]{3.png}
    \end{figure}
    \begin{center}
        \small \textit{Фигура 1: Електронно-микроскопска снимка на трансмон. На електрическата схема отдолу се вижда системата кондензатор-SQUID елемент (Superconducting QUantum Interference Device).\\
        Източник: \href{https://arxiv.org/pdf/2106.11352.pdf}{arXiv:2106.11352, 2021}} \cite{roth2021introduction}
    \end{center}
    

    Токоносителите във връзките на Джозевсън не са единични електрони, а представляват т.нар. Купърови двойки (двойки на Bardeen-Cooper-Schrieffer).
    Купъровите двойки са пример за композитни бозони, съставени от два взаимодействащи си електрона. Квантово-механично връзката помежду им грубо се обяснява
    чрез електрон-фононното взаимодейсвие, където фононите представляват колективното движение на положително заредената кристална решетка
    в свръхпроводника. От друга страна бозонната структура на Купъровите двойки позволява те да заемат едно и също квантово състояние (кондензат на
    Бозе-Айнщайн), което води до свръхпроводимост.\\
    Контролирането и прочитането на информацията в кюбитите обикновено става чрез използването на електромагнитно лъчение в микровълновия спектър.
    Чрез подходящ подбор на интензивността на лъчението както и познаването на резонансната честота на трептене на Джозевсъновите връзки, същите
    могат да бъдат използвани като осцилатори, които да преминават от основно във възбудено състояние и обратно.
    Важно условие за правилното функциониране на квантовия бит обаче е наличието на температури близки до $0 K$, което налага използването на
    специален тип системи за охлаждане.

    \section{Охладителна система}

    \nocite{refrigeratorsuperconducting}
    Достигането на криогенни температури при този тип квантови системи се осъществява чрез т.нар. охладителни системи с разреден хладилен агент
    (dilution refrigerators), които осигуряват необходимата за правилната работа на кюбитите температура. Работното вещество
    се състои от два изотопа на хелия -- $^{3}He$ и $^{4}He$, като след смесване и по-нататъшно изпомпване на $^{3}He$ системата достига до температурни стойности
    по-ниски от петнадесет миликелвина ($\le 15 \times 10^{-3} K$). Устройството на системата е такова, че температурата намалява поетапно с преминаването все по-надолу по конструкцията на охладителната система.
    Квантовия процесор се закрепя в най-ниската точка, където е и най-ниската температура. Това задържа кюбитите по-дълго време в стабилно състояние и позволява реализирането на контролирани квантови преходи.\\
    За да разберем защо двата изотопа на хелия се смесват спонтанно един с друг, нека си представим ситуацията на охладени $^3He$ и $^4He$ до температура $0$К. Заради принципа за
    неопределеност на Хайзенберг $\left(\Delta x \Delta p = \frac{\hbar}{2}\right)$, дори при изключително ниски температури атомите на хелия се движат в рамките на определена област от пространството,
    пряко зависеща от масата на частиците. Хелият е инертен газ, което означава, че единственото възможно взаимодействие между двата изотопа, съставящи хладилният агент на охладителната
    система е вследствие на наличието на слаби Ван дер Ваалсови сили. Тъй като $^3He$ е с около $25\%$ по-лек от $^4He$ част от атомите на по-лекия изотоп ще се разтворят в по-тежкия поради по-късото разстояние във
    връзката $^3He \leftrightarrow ^4He$, отколкото тази на $^3He \leftrightarrow ^3He$. Този процес е свързан с топлообмен от околната среда към сместа. В охладителните системи с разреден хладилен агент именно на
    границата между двете течни фази се поставя обекта, чиято температура трябва да бъде понижена.

    \begin{figure}[H]
        \centering
        \includegraphics[width=230pt]{18.png}
    \end{figure}
    \begin{center}
        \small \textit{Фигура 2: Различните амплитуди на трептене на $^3He$ и $^4He$ благоприятстват връзката между двата различни изотопа.}
    \end{center}

    Най-простото обобщение на една такава охладителна система може да се даде чрез U-образна тръба от едната страна на която са наляти двете течни фази на $^3He$ и $^4He$. Тъй като $^3He$ е по-лек, то той
    ще изплува на повърхността като в началото ще се образува ясна граница между двата изотопа. Заради ефектите, обяснени по-рано част от по-лекия изотоп ще започне да се разтваря в по-тежкия до достигане
    на равновесна концентрация (около $6,4\%$). Ако от другата страна на тръбата поставим нагревател с помпа ще създадем условия за отделяне на $^3He$ от разтвора, тъй като той има по-ниска температура на
    кипене, а помпата ще послужи за физическото отделяне на атомите от сместа. Така разтворът ще започне да обеднява от страна на по-лекия изотоп, което на своя страна ще доведе до ново разтваряне на границата
    между двете течни фази. Процесът може да продължи вечно като при всяко ново разтваряне на границата се поглъща топлина от околната среда и всеки образец поставен в близост ще започне да се охлажда.\\
    Разбира се устройството на същинските охладителни апарати е по-сложно. То се състои основно от пет части: циркулационна система, топлообменник, дестилатор, смесителна камера и фононен изолатор. \cite{hall1966helium}\\

    \begin{figure}[H]
        \centering
        \includegraphics[width=230pt]{1.png}
    \end{figure}
    \begin{center}
        \small \textit{Фигура 3: Охладителната система Quantum System One на IBM.\\
        Източник: \href{https://www.ibm.com/quantum-computing/systems}{Commanding the power of nature with IBM Quantum systems}}
    \end{center}

    \vspace{3mm}

    \normalsize{\textit{\underline{Циркулационна система}}}\\

    Циркулационната система служи за преноса на отделните изотопи на хелия и тяхната смес до останалите части на апарта. В конструкцията ѝ са включени също помпи, спомагащи преноса на
    изотопите и хелиевата смес до различни части на охладителната система както и импеданси, водещи към дестилатора и смесителната камера. Импедансите представляват капилярки с
    високо съпротивление при протичане на хладилния агент. По време на протичането си през отделните части на охладителната машина, хладилния агент се охлажда поетапно чрез серия от
    топлообменници до достигане на смесителната камера. \cite{craig2004hitchhiker}

    \vspace{3mm}

    \normalsize{\textit{\underline{Топлообменник}}}\\

    Топлообменниците са устройства, способстващи преноса на топлина между отделни части на охладителната система. По пътя на чистата фаза на хелия към смесителната камера например,
    се намира топлообменник, който позволява охлаждането на чистия хелий от изпомпваната към дестилатора хелиева смес, като с това топлината на отделните изотопи намалява още повече,
    което води до по-добър охлаждащ ефект при смесването на изотопите в смесителната камера.

    \vspace{3mm}

    \normalsize{\textit{\underline{Смесителна камера}}}\\

    След преминаване през останалите етапи на предварително охлаждане, двата изотопа навлизат в смесителната камера ($^{3}He$ "изплува" над $^{4}He$ поради по-малката си маса)
    където се случва спонтанно разтваряне на $^{3}He$ в $^{4}He$. Тъй като хелия е инертен газ, свързване между отделните атоми се осъществява от слаби Ван-дер-Ваалсови сили на привличане,
    които са по-здрави при $^{3}He \leftrightarrow {^{4}He}$, отколкото между два $^{3}He$ атома. Поради тази причина на границата между двете фази в смесителната камера преференциално се случва разтваряне
    на по-лекия изотоп в по-тежкия. Това води до увеличаване на ентропията на системата съпроводено с поглъщане на топлина, което води до охлаждане на всичко поставено в термичен
    контакт с мястото на образуваната граница между двата изотопа. Образуват се т.нар. обогатена и разредена фаза, като след това разредената фаза се насочва към дестилатора.

    \vspace{3mm}

    \normalsize{\textit{\underline{Дестилатор}}}\\

    При навлизане на разредената фаза в дестилатора тя се нагрява, при което $^{3}He$ се изпарява (поради по-ниската си температура на кипене) и бива изпомпан извън дестилатора, в циркулационната
    система. В останалата част от сместа, сега почти изцяло $^{4}He$ от едната страна, се образува осмотично налягане, което придвижва разтворения $^{3}He$ към дестилатора. Съществува и метод, чрез
    който тази смес се пренасочва през свръхфлуидна капилярка (позволяваща преминаването единствено на свръхфлуидния $^{4}He$) обратно към смесителната камера, където да послужи за ново разтваряне
    на $^{3}He$ и ново понижаване на температурата. \cite{martin2010closed}

    \vspace{3mm}

    \normalsize{\textit{\underline{Фононен изолатор}}}\\

    За по-добра топлинна изолация се използват т. нар. фононни изолатори, които служат като буфери на молекулни трептения. С други думи, вследствие работата на охладителната система могат да се появят
    вибрации в частите, отговарящи за охлаждането на образеца, което води до нежелано повишаване на температурата. Това може да бъде избегнато с помощта на устройства, компенсиращи излишните вибрации,
    с което се постига съответната изолация.

    \begin{figure}[H]
        \centering
        \includegraphics[width=230pt]{2.png}
    \end{figure}
    \begin{center}
        \small \textit{Фигура 4: Схематично представяне на охладителна система, работеща с изотопите $^{3}He/^{4}He$.\\
        Източник: \href{https://link.springer.com/content/pdf/10.1007/s10909-011-0373-x.pdf}{Journal of Low Temperature Physics, 164(5):179-236, 2011}}\cite{de2011basic}
    \end{center}

    От казаното дотук става ясно, че за правилната работа на свръхпроводниковия квантов компютър от особена важност е наличието на добра изолация и изключително ниски температури. С увеличаването на изчислителната мощ и броя
    кюбити в квантовите процесори тази задача става все по-трудна, което налага създаването на все по-нови и мощни охлаждащи системи.

    \section{Чипове и квантово превъзходство}

    Теоретично, квантов чип с над 50-70 взаимодействащи си кюбита е способен да решава задачи, невъзможни за решаване дори от най-мощния класически суперкомпютър (или поне невъзможни за решаване в обозримо бъдеще).
    Това предполагаемо свойство на квантовите компютри, познато като "квантово превъзходство" е една от основните мотивации за създаване на изчислителни устройства, разчитащи на квантовите ефекти за извършване на
    своята дейност. На практика обаче откриването на такава задача е трудно, тъй като съществуват редица пречки, които трябва да бъдат превъзмогнати. Например не винаги е ясно, че дадената задача наистина е решима
    единствено от квантов компютър, тъй като е възможно да бъде създаден ефективен алгоритъм за класически компютър в бъдеще. Съществуват и проблеми за отделянето на правилното решение (error correction), породени от
    некохерентности и шум, създаване на ефективни алгоритми, съвместими със законите на квантовата механика за решаване на определена задача и др.\\
    Въпреки това вече са създадени редица технологии за конструиране на квантови процесори както и алгоритми за по-дълги времена на кохерентност и елиминиране на грешките при изчисленията, както и евентуални техни подобрения.
    Например предишни поколения квантови процесори използват трансмони, организирани в двумерна правоъгълна решетка като по този начин всеки кюбит (с изключение на крайните) е свързан с четири свои съседа с цел по-голяма
    прецизност при измерванията на зададеното състояние. Такава структура обаче води и до така наречения "crosstalk" между най-близките съседи, който на своя страна повишава грешката при измерване. Сред квантовите системи вече
    все повече започва да се налага хексагоналното организиране, което намалява броя най-близки съседи на три, а с това и съответния "crosstalk".\\
    През 2019 г. Гугъл обявява че техният Sycamore процесор (53 кюбита) е завършил изчисление за 200 секунди, което на класически суперкомпютър (Summit) би отнело около 10 000 години. Въпреки че това твърдение не бе
    напълно потвърдено (по-късното становище на IBM е че техният суперкомпютър може да реши предложената задача за 2.5 дни), процесорът на Гугъл е бил използван за решаване на редица сложни задачи като химическа симулация,
    използваща методът на Хартри-Фок, решаване на квантовата задача за локализиране на конфигурация от частици със спин-нагоре и спин-надолу за дискретен времеви кристал и др.\\
    По-късно китайският свръхпроводников квантов компютър Zuchongzhi също демонстрира квантово превъзходство, използвайки сапфирен чип с 66 кюбита и 110 връзки. Проведените симулации в статия от юни 2021 г. \cite{wu2021strong} 
    показват значително повишаване на изчислителната мощ (2-3 порядъка) в сравнение с демонстрираната от Sycamore през 2019 г., разграничавайки още повече възможностите на квантовите компютри в сравнение с класическите машини.\\

    \begin{figure}[H]
        \centering
        \includegraphics[width=230pt]{8.jpg}
    \end{figure}
    \begin{center}
        \small \textit{Фигура 5: Компютърен модел на квантовия процесор Zuchongzhi. Показани са двата чипа, единият от които съдържа свързаните в правоъгълна решетка кюбити. Другият съдържа контролната шина както и отчитащите елементи.\\
        Източник: \href{https://journals.aps.org/prl/abstract/10.1103/PhysRevLett.127.180501}{Physical review letters, 127(18):180501, 2021}}\cite{wu2021strong}
    \end{center}

    Друг лидер в конструирането на квантови чипове е IBM, които за няколко години успяват да демонстрират изключителен напредък в развитието на своите системи (охлаждане и брой кюбити в чип). Настоящият 127 кюбитов чип Eagle
    е най-мощният в света. Структурата му дава възможност за по-нататъшно увеличаване на броя кюбити, което е планирано да се случи през следващите няколко години. В допълнение, разширяването на системите след границата от 1000
    кюбита се очаква да се осъществи с помощта на допълнителни структури като връзки тип чип-чип (chip-to-chip couplers), позволяващи директна комуникация между два кюбита намиращи се в два отделни чипа. Връзки, осъществяващи
    класическа комуникация между отделни чипове или цели чипсетове, отделните елементи на които са свързани чрез чип-чип връзки, също се планират до края на 2025 г. Въпреки че подобно паралелно свързване е свързано със значително
    забавяне по време на изчисленията, то ще доведе до по-голяма и бърза мащабируемост на системите, което ще даде възможност на потребителите да използват много по-голям брой кюбити за осъществяване на своята дейност.

    \begin{figure}[H]
        \centering
        \includegraphics[width=430pt]{7.png}
    \end{figure}
    \begin{center}
        \small \textit{Фигура 6: Графично представяне на развитието на чиповете на IBM от гледна точка на броя кюбити в един чип (сините квадрати).} \textbf{А)} \textit{Hummingbird с 65 кюбита, създаден през 2020 г.,} \textbf{Б)}
        \textit{настоящия Eagle с 127 кюбита (2021 г.)} \textbf{В)} \textit{Osprey: 433 кюбита (2022 г.) и} \textbf{Г)} \textit{Condor: 1121 кюбита (2023 г.)}\\
    \end{center}

    \begin{figure}[H]
        \centering
        \includegraphics[width=330pt]{9.png}
    \end{figure}
    \begin{center}
        \small \textit{Фигура 7: Структура тип чип-чип, използвана от IBM за паралелизиране и допълнително увеличаване на общия брой кюбити, участващи в един изчислителен процес.}\\
    \end{center}

    До сега обърнахме внимание най-вече на хардуерната част от изграждането на един квантов компютър и методите за реализация на ефективна система, проявяваща квантови свойства на макрониво.
    Фокусът в следващите глави ще бъде върху контрола над преходите между състоянията в единичен кюбит, притежаващ поведение на хармоничен осцилатор.