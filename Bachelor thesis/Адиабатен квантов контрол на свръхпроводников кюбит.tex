\documentclass[12pt]{report}

\usepackage{geometry}
\geometry{
    a4paper,
    total={170mm,257mm},
    left=30mm,
    top=20mm,
}
\usepackage{amssymb}
%\usepackage[superscript,biblabel]{cite}
\usepackage{eso-pic}
\usepackage{import}
\usepackage{amsmath}
\usepackage{cite}
\usepackage{subcaption}
\usepackage{mathtools}
\usepackage{physics}
\usepackage{float}
\usepackage{color}
\usepackage{authblk}
\usepackage[colorlinks=black,citecolor=black,urlcolor=blue,bookmarks=false,hypertexnames=true]{hyperref}
\usepackage{latexsym}
\usepackage{hyperref}
\usepackage{url}
\usepackage[utf8]{inputenc}
\usepackage{graphicx}
\usepackage{footnotebackref}
\usepackage{setspace}
\usepackage[labelformat=empty]{caption}
\usepackage[T2A]{fontenc}
\usepackage[bulgarian]{babel}
\selectlanguage{bulgarian}

\author{}

\title{}
\date{}

\begin{document}

    \begin{titlepage}
        \begin{center}
            \vspace*{3.5cm}
            \Huge\textbf{Дипломна работа}\\
            \vspace{0.5cm}
            \Large
            За придобиване на образователно-квалификационна степен \textbf{бакалавър}\\
            \vspace{1cm}
            \hrule
            \vspace{0.45cm}
            \Huge
            \textbf{Адиабатен квантов контрол на свръхпроводников кюбит}\\
            \vspace{0.45cm}
            \hrule
            \vspace{3cm}
            \Large
            \textit{Представена от:}\\
            Ивайло Иванов  Иванов\\
            Квантова и Космическа Теоретична Физика\\
            Факултетен № 13131\\
            \vspace{0.3cm}
            Научен ръководител: акад. Николай Витанов\\
            \vspace{5cm}
            \begin{figure}[htp]
                \centering
                \includegraphics[width=330pt]{Untitled.png}
            \end{figure}

        \end{center}
    \end{titlepage}

        \large
        \tableofcontents
        \thispagestyle{empty}
        \newpage

    \clearpage
    \pagenumbering{arabic}

        \chapter*{Резюме}
    \small \textit{Настоящата работа разглежда някои прложения на адиабатните методи за контрол над преходите на свръхпроводников кюбит
    и по-нататъшното им реализиране в квантови системи като тези на IBM и Гугъл. Първоначално е предоставено кратко изложение на принципа
    на действие на свръхпроводниковия кюбит, неговата реализация като част от цяла система при конструирането на квантови чипове и необходимата
    апаратурата за конструирането на работещо квантово устройство. В края на тази глава са представени едни от най-добре развитите квантови устройства,
    принципът им на работа, както и някои интересни планове за развитието на технологията на квантовите компютри в обозримо бъдеще. Изведени са вече
    познатите модели на система с две нива -- резонансни и адиабатни решения без и с пресичане на нивата, като поотделно е обърнато внимание на отделните
    видове пресичания при последните, реализиращи познатите в литературата квантови гейтове чрез различни крайни конфигурации на заселеност. Описани са и
    системите с три нива, като отново е разгледано резонансното решение. Тук то се разделя на два подвида -- едно- и двуфотонен резонанс. Отделено
    е специално внимание на системите с два детюнинга, като спецификата тук се състои в обосноваването на предположението, че най-високото по енергия ниво може
    да бъде изолирано от останалите две, като по този начин последните формират система с две нива, която ефективно да може да се използва за изчислителна единица
    (кюбит). Разгледани са и някои алтернативни методи за квантов контрол като бързи композитни гейтове с висока прецизност и DRAG и GRAPE алгоритмите,
    регулиращи външното поле за постигане на по-добра защита от изтичане на заселеност към по-високи състояния (leakage errors).}\vspace{10mm}

    \import{./}{Chapter 1.tex}

    \import{./}{Chapter 2.tex}

    \import{./}{Chapter 3.tex}

    \import{./}{Chapter 4.tex}

    \import{./}{Chapter 5.tex}
    
    \import{./}{Aknowledgements.tex}

            \newpage
            \thispagestyle{empty}
            \bibliographystyle{unsrt}
            \bibliography{citation}

\end{document}
  