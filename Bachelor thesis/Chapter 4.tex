\chapter{Алтернативни методи за квантов контрол} \nocite{gambetta2011analytic}

    Съществуват методи, с помощта на които се постига понижаване на вероятността за преход към по-високи състояния и изолиране на първите две нива в свръхпроводниковите кюбити.
    В тази глава са предсавени накратко някои подходи, различни от адиабатния, чрез които се постигат големи скорости на преход между отделните нива при слабо ахармоничен кюбит, както
    и значително намаляване на грешките, предизвикани от некохерентности, изтичане на заселеност към по-високи нива и др.

    \section{Метод на композитните импулси}

    Един от най-новите подходи за подобряване на точността на свръхпроводниковите кюбити се състои в прилагането вместо на един, на множество контролни пулсови сигнали с подходящо избрани
    честоти на Раби и относителни фази \cite{torosov2022fast}. Накратко методът се състои в следното: \\
    Както е добре известно трансмонът може да се опише приближено като ахармоничен осцилатор (контролиран от външно микровълново поле) с енергитични нива:

    \begin{equation}
        \omega_n = \left(\omega + \frac{\delta}{2}\right)n - \frac{\delta}{2}n^2 \quad,
    \end{equation}

    където $\omega$ е собствената честота на кюбита, а $\delta$ е ахармоничността. След преминаване към картина на взаимодействие и прилагане на RWA, Хамилтонианът на системата приема
    следния вид:

    \begin{equation} \label{4.2}
        \hat{H} = \frac{\hbar}{2}
        \begin{bmatrix}
            0 & \Omega & 0 \\
            \Omega^* & 0 & \sqrt{2}\Omega \\
            0 & \sqrt{2}\Omega^* & -2\delta
        \end{bmatrix}
    \end{equation}

    Тук $\Omega$ е честотата на Раби, задаваща силата на свързване между полето и кюбита (тъй като в общия случай честотата на Раби е комплексно число с $\Omega^*$ е обозначена комплексно
    спрегнатата стойност на $\Omega$). Използвайки така получения Хамилтониан можем да получим пропагатора, задаващ еволюцията на системата $\hat{U} = e^{-i\hat{H}t}$. Прилагането на метода
    на композитните импулси се състои в избирането на поредица от импулси с подходящи честоти на Раби и относителни фази. Това води до появата на нов пропагатор, представляващ произведение
    от отделните пропагатори за всеки импулс по отделно:

    \begin{equation}
        \hat{U}^{(N)} = \hat{U}(\Omega_N,\phi_N)\dots \hat{U}(\Omega_2, \phi_2)\hat{U}(\Omega_1,\phi_1)
    \end{equation}

    където $\hat{U}(\Omega_k,\phi_k)$ е съответния единичен пропагатор, отговарящ на Хамилтониана (\ref{4.2}), а честотата на Раби $\Omega_k e^{i\phi_k}$ е отместена с фаза $\phi_k$. Точно
    $\Omega_k$ и $\phi_k$ се използват като контролни параметри за постигането на пропагатор $\hat{U}^{N}$, който минимизира изтичането на заселеност към състояния извън базиса $\ket{0}$ и $\ket{1}$.
    Освен подобряване на точността, при подходящо избрани контролни параметри може да се постигне и възможност за прилагане на по-кратки контролиращи импулси, което води до значително повишаване на
    скоротта на гейтовете, приложени върху съответния кюбит.

    \begin{figure}[H]
        \centering
        \includegraphics[width=330pt]{6.png}
    \end{figure}
    \begin{center}
        \small \textit{Фигура 16: Графиката показва заселеностите на състоянията $\ket{0}$ и $\ket{1}$ след единичен $\pi$ пулс (пунктир) и след поредица от 6 пулса с подходящо избрани честота на Раби и относителна фаза (непрекъсната линия).\\
        На вмъкнатата графика са показани съответните изтичания към по-високото състояние $\ket{2}$.\\
        Източник: \href{https://arxiv.org/pdf/2205.04155.pdf}{arXiv:2205.04155, 2022}} \cite{torosov2022fast}
    \end{center}

    \section{DRAG (Derivative Removal by Adiabatic Gate) и \\
    GRAPE (Gradient Ascent Pulse Engineering)}

    Съществуват някои аналитични подходи, чрез които отново се постига добър контрол над преходите между състоянията в свръхпроводниковия кюбит. Един от тях е така нареченият DRAG алгоритъм, който се състои в добавянето на втори контролен
    параметър, пропорционален на производната по времето на първия. По същество DRAG импулса представлява Гаусов импулс с допълнителна компонента, служеща за намаляване на честотния спектър на нормалния Гаусиан за прехода $\ket{1} \rightarrow \ket{2}$,
    по този начин намалявайки вероятността за преход към състояние $\ket{2}$.\\
    В софтуерния пакет Qiskit (IBM) например DRAG алгоритъма се задава по следния начин \cite{drag}:

    \begin{align*}
        & g(x) = e^{-\frac{1}{2}\frac{(x-\frac{\Delta t}{2})^2}{\sigma^2}} \\
        & g'(x) =A \times \frac{g(x)-g(-1)}{1-g(-1)} \\
        & f(x) = g'(x) \times \left(1+1j\times\beta\times \left(-\frac{x-\frac{\Delta t}{2}}{\sigma^2}\right)\right) \quad , \quad 0\le x <\Delta t
    \end{align*}

    Тук $g(x)$ е обикновената Гаусова функция, $g'(x)$ е корекцията на Гаусиана. $f(x)$ представлява крайния вид на DRAG функцията. Параметрите $\Delta t, \, A, \, \beta, \, \sigma$ са съответно времетраенето на импулса по отношение на периода на измерване, амплитудата на обвивката на импулса,
    амплитудата на корекция и широчината на Гаусовата функция.

    \begin{figure}[H]
        \centering
        \includegraphics[width=370pt]{10.png}
    \end{figure}
    \begin{center}
        \small \textit{Фигура 17: Графика, представяща зависимостта на амплитудата от параметъра х. От сравняването на двете графики се вижда, че DRAG модулирания сигнал (в жълто) притежава по-голяма устойчивост на максимума от обикновения Гаусов импулс (синьо).}\\
    \end{center}

    Методът GRAPE \cite{motzoi2009simple} използва градиента на еволюционния оператор на приложения импулс с цел внасяне на поправка и подобряване на ефективността на прехода.
    За да разберем как точно се стига до този алгоритъм нека разгледаме най-простия метод на трансфер между ермитови оператори в отсъствие на релаксационни процеси.\\
    Нека състоянието на системата се характеризира от операторът на плътността $\rho(t)$, а уравнението на движение е:
    \begin{equation}
        \dot{\rho}(t) = -\textit{i}\left[\left(H_0 + \sum_{j=1}^s u_j(t)H_j\right),\rho(t)\right],
    \end{equation}
    тук $H_0$ е свободният еволюционен Хамилтониан, $H_j$ са Хамилтонианите на контролните полета, а $u(t) = (u_1(t), u_2(t), \dots , u_s(t))$ е контролният вектор (отговорен за промяната в амплитудите при прилагане на алгоритъма). Сега задачата ни се свежда до намиране на
    подходящите стойности на $u_j(t)$, водещи оператора на плътността $\rho(0) = \rho_0$ до някаква желана стойност $\rho(T)$ за определен момент $T$. Тази желана стойност всъщност се води от факта, че операторът на плътността трябва да съвпада максимално с целеви оператор, който
    тук ще обозначим с $C$.
    За ермитови оператори това може да се измери с произведението $\braket{C}{\rho(T)} = Tr\left\{C^\dagger \rho(t)\right\}$. Следователно показателят на действие се задава с формулата:
    \begin{equation}
        \Phi_0 = \braket{C}{\rho(T)}.
    \end{equation}
    За опростяване на по-нататъшните разглеждания ще разделим времето $T$ на $N$ равни интервала $\Delta t=\frac{T}{N}$, такива че амплитудата $u_j(k)$, съответстваща на всеки интервал ще бъде константна. Така времевата еволюция на системата за един интервал ще се задава чрез пропагатора:
    \begin{equation} \label{4.6}
        U_k = e^{-\textit{i}\Delta t \left(H_0 + \sum_{j=1}^s u_j(k)H_j \right)},
    \end{equation}
    а оператора на плътността за време $t=T$ ще бъде:
    \begin{align}
        & \rho(T) = U_N \dots U_1 \rho_0 U_1^\dagger \dots U_N^\dagger\\
        \Rightarrow &\Phi_0 = \braket{C}{U_N \dots U_1 \rho_0 U_1^\dagger \dots U_N^\dagger}.
    \end{align}
    Използвайки свойствата на вътрешното произведение, можем да пренапишем последното равенство така:
    \begin{equation} \label{4.9}
        \Phi_0 = \bigl<\underbrace{U_{k+1}^\dagger \dots U_N^\dagger CU_N\dots U_{k+1}}_{\lambda_k}|\underbrace{U_k\dots U_1 \rho_0 U_1^\dagger \dots U_k^\dagger}_{\rho_k}\bigr>,
    \end{equation}
    където с $\rho_k$ сме отбелязали оператора на плътността $\rho(t)$ за време $t=k\Delta t$, а с $\lambda_k$ -- обратно-пропагирания оператор $C$ за същото време.\\
    Нека сега разгледаме как ще се промени $\Phi_0$ след въвеждане на пертурбация в амплитудата $u_j(k) + \delta u_j(k)$. От уравнение (\ref{4.6}) се вижда, че промяната на $U_k$ до първа степен на пертурбацията $\delta u_j(k)$ е:
    \begin{equation} \label{4.10}
        \delta U_k = -\textit{i}\Delta t\delta u_j(k)\overline{H}_jU_k \, \, \, \, ,
    \end{equation}
    където $\overline{H}_j \Delta t = \int_{0}^{\Delta t} U_k(\tau)\mathcal{H}_j U_k(-\tau) \,d\tau $, а $U_k(\tau) = e^{-\textit{i}\tau\left(H_0 + \sum_{j=1}^s u_j(k)H_j\right)}$.\\
    За малки стойности на $\Delta t \, \left(\Delta t \ll \left|\left|H_0 + \sum_{k=1}^s u_j(k)H_j\right|\right|^{-1}\right)$ и използвайки уравнения (\ref{4.9}) и (\ref{4.10}) достигаме до:
    \begin{equation}
        \frac{\delta\Phi_0}{\delta u_j(k)} = -\braket{\lambda_k}{\textit{i}\Delta t\left[H_j,\rho_k\right]}
    \end{equation}
    И така, за обикновения GRAPE алгоритъм дефинираме следните стъпки:
    
    \begin{enumerate}
        \item правим предположение за стойността на началните контролни параметри $u_j(k)$;
        \item започвайки от $\rho_0$ изчисляваме всички $\rho_k = U_k\dots U_1\rho_0U_1^\dagger\dots U_k^\dagger$, където $k\le N$;
        \item започвайки от $\lambda_N = C$ пресмятаме всички $\lambda_k = U_{k+1}^\dagger\dots U_N^\dagger CU_N\dots U_{k+1}$, за $k\le N$;
        \item намираме $\frac{\delta \Phi_0}{\delta u_j(k)}$ и обновяваме $s\times N$ контролните амплитуди $u_j(k)$ според уравнението $u_j(k) \rightarrow u_j(k) + \varepsilon \frac{\delta \Phi_0}{\delta u_kj(k)}$;
        \item с така получените данни се връщаме обратно на стъпка 2.
    \end{enumerate}

    \begin{figure}[H]
        \centering
        \includegraphics[width=250pt]{13.png}
    \end{figure}
    \begin{center}
        \small \textit{Фигура 18: Графично представяне на амплитудата $u_j(t)$ за времеви период $T$, разделен на $N$ равни интервала. Вертикалните стрелки представляват градиентите $\frac{\delta\Phi_0}{\delta u_j(k)}$, показващи колко трябва да бъде модифицирана амплитудата за да се подобри функцията $\Phi_0$.\\
        Източник: \href{https://www.sciencedirect.com/science/article/abs/pii/S1090780704003696}{Journal of magnetic resonance, 172(2):296-305, 2005}} \cite{khaneja2005optimal}
    \end{center}